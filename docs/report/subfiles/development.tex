Before we are going into the details of the
development process let's first look at
the high level overview of what we created
(note that you can click on the Mst name to see
whole documentation of it).

\begin{description}
  \item[\href{https://github.com/PraxTube/chess-ai/tree/master/docs/milestones/1-dummy-AI}{Mst1 - Dummy AI:}] \hfill
    \begin{itemize}
      \item Chess Backend
      \item Dummy AI (minimax)
      \item Basic Evaluation (Material only)
    \end{itemize}
  \item[\href{https://github.com/PraxTube/chess-ai/tree/master/docs/milestones/2-basic-AI}{Mst2 - Basic AI:}] \hfill
    \begin{itemize}
      \item Improve Backend
      \item Alpha-Beta Tree Search
      \item Improved Evaluation (PeSTO)
      \item Better time management
    \end{itemize}
  \item[\href{https://github.com/PraxTube/chess-ai/tree/master/docs/milestones/3-advanced-AI}{Mst3 - Advanced AI:}] \hfill
    \begin{itemize}
      \item Restructured Chess Backend
      \item Speed up Evaluation (through numpy)
      \item Improve move ordering
      \item Include King of the Hill in evaluation
      \item Restructure internal debug info
    \end{itemize}
  \item[\href{https://github.com/PraxTube/chess-ai/blob/documentation-milestone-four/docs/milestones/4-optimized-AI/README.md}{Mst4 Optimized AI:}] \hfill
    \begin{itemize}
      \item Improve evaluation
      \item Use Monte Carlo Tree Search
      \item Implement PVS/negamax
      \item Add Nullsearch
    \end{itemize}
\end{description}

\pagebreak

\subsection{Mst1 Dummy AI}


