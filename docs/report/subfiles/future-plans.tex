As we reflect on the journey of developing our chess AI, we are excited about the potential directions for future work. The completion of this project marks not an end, but the beginning of a new phase of exploration and innovation.

We are interested in exploring the use of neural networks \cite{david2016deepchess} for position evaluation. Traditional chess AIs, including ours, use handcrafted evaluation functions to assess the desirability of a given board position. However, these functions are inherently limited by our understanding of the game. Neural networks, on the other hand, have the potential to uncover subtle patterns and strategies that are beyond human comprehension. By training a neural network on a large dataset of high-level chess games, we could potentially develop a more nuanced and powerful evaluation function.

Another area for future work is the optimization of our AI's search algorithm. While our current implementation of the PVS/negamax algorithm has proven effective, there is always room for improvement. For instance, we could explore more sophisticated move ordering strategies to increase the efficiency of the search. We could also investigate the use of more advanced pruning techniques to cut off unproductive branches of the search tree.

Finally, we plan to continue refining the backend of our chess system. While the backend has served us well so far, we believe that there are opportunities for further optimization and enhancement. For example, we could implement support for additional chess variants, or develop a more user-friendly interface for interacting with the AI.

In conclusion, while we are proud of what we have achieved with our chess AI, we are even more excited about what the future holds. We look forward to continuing our journey.
