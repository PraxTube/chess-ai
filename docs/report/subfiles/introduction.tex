Starting the development of a chess AI from the ground up presents a considerable challenge.
Not only will you need to write a whole chess backend, you will also
need to implement the AI features. A significant hurdle in this process is the development of a bug-free chess backend, coupled with the need for rigorous testing of the AI to ensure that the integration of new features enhances its performance.

Our group chose to use Python for the whole project driven by our collective familiarity and proficiency with this language. 
his choice, however, was not without its trade-offs. While Python facilitates rapid prototyping, it is also inherently slow and can be susceptible to errors.
Some team members expressed an interest in exploring the use of Rust, a language known for its performance and memory safety. However, in retrospect, the adoption of Rust \cite{je2020scientists} might have led to the termination and frustration in the early stages of the project, given the extensive workload associated with the development of the chess engine alone. That being said, some of us have since gained experience with Rust and, in light of this, would now prefer to use it over Python in a project like this.

Throughout the course of this project, we utilized Git and GitHub for version control, irrespective of our individual programming language preferences. This approach facilitated a seamless and efficient workflow with minor merge conflicts while working simultaneously on this project.

This project represented our first substantial engagement with AI development. Our initial aim was to create a basic AI, a goal we not only achieved but surpassed. The final product, while not the most powerful due to our use of Python, exceeded our expectations and provided us with valuable learning experiences.

In the following sections, we will dive into the specifics of our AI's creation, discussing the methods we employed, the reasoning behind our decisions and the lessons we learned. We will also provide a reflective analysis of the project as a whole.
