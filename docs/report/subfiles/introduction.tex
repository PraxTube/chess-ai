Creating a chess AI from scratch is quite a challenging undertaking.
Not only will you need to write a whole chess backend, you will also
need to implement the AI features. One of the major issues here
is to write a chess backend without any bugs and to test your
AI properly to make sure the features you add actually make it
play better.

Our group chose to use Python for the whole project given
that that is what we were most familiar with.
The obvious trade-off here is of course that it's easy
to prototype but painfully slow and very error prone.
I personally would have liked to try to use Rust,
though in hindsight we would have probably abandoned
the project if we had used Rust simply because
the chess engine alone was so much work.
On the other side I have acquired some Rust experience
now and if I were to write the chess engine
(or something of a similar level) I would probably
go with Rust.

Regardless of our programming language, for version control
we obviously used git and to share our code base we used github.
The overall workflow here was pretty smooth.
