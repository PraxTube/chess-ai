Creating a chess AI from scratch is quite a challenging undertaking.
Not only will you need to write a whole chess backend, you will also
need to implement the AI features. One of the major issues here
is to write a chess backend without any bugs and to test your
AI properly to make sure the features you add actually make it
play better.

Our group chose to use Python for the whole project given
that that is what we were most familiar with.
The obvious trade-off here is of course that it's easy
to prototype but painfully slow and very error prone.
I personally would have liked to try to use Rust,
though in hindsight we would have probably abandoned
the project if we had used Rust simply because
the chess engine alone was so much work.
On the other side I have acquired some Rust experience
now and if I were to write the chess engine
(or something of a similar level) I would probably
go with Rust.

Regardless of our programming language, for version control
we obviously used git and to share our code base we used github.
The overall workflow here was pretty smooth.

This was pretty much the first \textit{real} AI project we
took on. Our goal was to at least get a basic AI done,
that was what we wanted to to reach at least.
Our final AI is actually fairly advanced for what we seeked
to accomplish. It's obviously not the strongest (
given that we are using python that isn't too surprising).
However we are very pleased with the end result and with what
we created and learned during this project.

In the following chapters we will go into more detail
into what, why and how we created our AI.
We will also reflect on all these things.
